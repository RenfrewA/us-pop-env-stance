% Options for packages loaded elsewhere
\PassOptionsToPackage{unicode}{hyperref}
\PassOptionsToPackage{hyphens}{url}
\PassOptionsToPackage{dvipsnames,svgnames,x11names}{xcolor}
%
\documentclass[
  letterpaper,
  DIV=11,
  numbers=noendperiod]{scrartcl}

\usepackage{amsmath,amssymb}
\usepackage{iftex}
\ifPDFTeX
  \usepackage[T1]{fontenc}
  \usepackage[utf8]{inputenc}
  \usepackage{textcomp} % provide euro and other symbols
\else % if luatex or xetex
  \usepackage{unicode-math}
  \defaultfontfeatures{Scale=MatchLowercase}
  \defaultfontfeatures[\rmfamily]{Ligatures=TeX,Scale=1}
\fi
\usepackage{lmodern}
\ifPDFTeX\else  
    % xetex/luatex font selection
\fi
% Use upquote if available, for straight quotes in verbatim environments
\IfFileExists{upquote.sty}{\usepackage{upquote}}{}
\IfFileExists{microtype.sty}{% use microtype if available
  \usepackage[]{microtype}
  \UseMicrotypeSet[protrusion]{basicmath} % disable protrusion for tt fonts
}{}
\makeatletter
\@ifundefined{KOMAClassName}{% if non-KOMA class
  \IfFileExists{parskip.sty}{%
    \usepackage{parskip}
  }{% else
    \setlength{\parindent}{0pt}
    \setlength{\parskip}{6pt plus 2pt minus 1pt}}
}{% if KOMA class
  \KOMAoptions{parskip=half}}
\makeatother
\usepackage{xcolor}
\setlength{\emergencystretch}{3em} % prevent overfull lines
\setcounter{secnumdepth}{5}
% Make \paragraph and \subparagraph free-standing
\ifx\paragraph\undefined\else
  \let\oldparagraph\paragraph
  \renewcommand{\paragraph}[1]{\oldparagraph{#1}\mbox{}}
\fi
\ifx\subparagraph\undefined\else
  \let\oldsubparagraph\subparagraph
  \renewcommand{\subparagraph}[1]{\oldsubparagraph{#1}\mbox{}}
\fi


\providecommand{\tightlist}{%
  \setlength{\itemsep}{0pt}\setlength{\parskip}{0pt}}\usepackage{longtable,booktabs,array}
\usepackage{calc} % for calculating minipage widths
% Correct order of tables after \paragraph or \subparagraph
\usepackage{etoolbox}
\makeatletter
\patchcmd\longtable{\par}{\if@noskipsec\mbox{}\fi\par}{}{}
\makeatother
% Allow footnotes in longtable head/foot
\IfFileExists{footnotehyper.sty}{\usepackage{footnotehyper}}{\usepackage{footnote}}
\makesavenoteenv{longtable}
\usepackage{graphicx}
\makeatletter
\def\maxwidth{\ifdim\Gin@nat@width>\linewidth\linewidth\else\Gin@nat@width\fi}
\def\maxheight{\ifdim\Gin@nat@height>\textheight\textheight\else\Gin@nat@height\fi}
\makeatother
% Scale images if necessary, so that they will not overflow the page
% margins by default, and it is still possible to overwrite the defaults
% using explicit options in \includegraphics[width, height, ...]{}
\setkeys{Gin}{width=\maxwidth,height=\maxheight,keepaspectratio}
% Set default figure placement to htbp
\makeatletter
\def\fps@figure{htbp}
\makeatother

\usepackage{booktabs}
\usepackage{longtable}
\usepackage{array}
\usepackage{multirow}
\usepackage{wrapfig}
\usepackage{float}
\usepackage{colortbl}
\usepackage{pdflscape}
\usepackage{tabu}
\usepackage{threeparttable}
\usepackage{threeparttablex}
\usepackage[normalem]{ulem}
\usepackage{makecell}
\usepackage{xcolor}
\usepackage{siunitx}

  \newcolumntype{d}{S[
    input-open-uncertainty=,
    input-close-uncertainty=,
    parse-numbers = false,
    table-align-text-pre=false,
    table-align-text-post=false
  ]}
  
\KOMAoption{captions}{tableheading}
\makeatletter
\makeatother
\makeatletter
\makeatother
\makeatletter
\@ifpackageloaded{caption}{}{\usepackage{caption}}
\AtBeginDocument{%
\ifdefined\contentsname
  \renewcommand*\contentsname{Table of contents}
\else
  \newcommand\contentsname{Table of contents}
\fi
\ifdefined\listfigurename
  \renewcommand*\listfigurename{List of Figures}
\else
  \newcommand\listfigurename{List of Figures}
\fi
\ifdefined\listtablename
  \renewcommand*\listtablename{List of Tables}
\else
  \newcommand\listtablename{List of Tables}
\fi
\ifdefined\figurename
  \renewcommand*\figurename{Figure}
\else
  \newcommand\figurename{Figure}
\fi
\ifdefined\tablename
  \renewcommand*\tablename{Table}
\else
  \newcommand\tablename{Table}
\fi
}
\@ifpackageloaded{float}{}{\usepackage{float}}
\floatstyle{ruled}
\@ifundefined{c@chapter}{\newfloat{codelisting}{h}{lop}}{\newfloat{codelisting}{h}{lop}[chapter]}
\floatname{codelisting}{Listing}
\newcommand*\listoflistings{\listof{codelisting}{List of Listings}}
\makeatother
\makeatletter
\@ifpackageloaded{caption}{}{\usepackage{caption}}
\@ifpackageloaded{subcaption}{}{\usepackage{subcaption}}
\makeatother
\makeatletter
\@ifpackageloaded{tcolorbox}{}{\usepackage[skins,breakable]{tcolorbox}}
\makeatother
\makeatletter
\@ifundefined{shadecolor}{\definecolor{shadecolor}{rgb}{.97, .97, .97}}
\makeatother
\makeatletter
\makeatother
\makeatletter
\makeatother
\ifLuaTeX
  \usepackage{selnolig}  % disable illegal ligatures
\fi
\IfFileExists{bookmark.sty}{\usepackage{bookmark}}{\usepackage{hyperref}}
\IfFileExists{xurl.sty}{\usepackage{xurl}}{} % add URL line breaks if available
\urlstyle{same} % disable monospaced font for URLs
\hypersetup{
  pdftitle={My title},
  pdfauthor={Renfrew Ao-Ieong},
  colorlinks=true,
  linkcolor={blue},
  filecolor={Maroon},
  citecolor={Blue},
  urlcolor={Blue},
  pdfcreator={LaTeX via pandoc}}

\title{My title\thanks{Code and data are available at:
\url{https://github.com/RenfrewA/us-pop-env-stance}}}
\usepackage{etoolbox}
\makeatletter
\providecommand{\subtitle}[1]{% add subtitle to \maketitle
  \apptocmd{\@title}{\par {\large #1 \par}}{}{}
}
\makeatother
\subtitle{My subtitle if needed}
\author{Renfrew Ao-Ieong}
\date{April 15, 2024}

\begin{document}
\maketitle
\begin{abstract}
First sentence. Second sentence. Third sentence. Fourth sentence.
\end{abstract}
\ifdefined\Shaded\renewenvironment{Shaded}{\begin{tcolorbox}[interior hidden, borderline west={3pt}{0pt}{shadecolor}, boxrule=0pt, enhanced, breakable, frame hidden, sharp corners]}{\end{tcolorbox}}\fi

\hypertarget{introduction}{%
\section{Introduction}\label{introduction}}

We are becoming more aware of the effects that humans have on the
environment. In the past \_\_\_\_\_ years, we have done
\_\_\_\_\_\_\_\_\_ to negatively impact the environment. To prevent this
on a large scale, it is up to the governments of countries. One of the
leading causes of pollution is the burning of fossil fuels to generate
electricity. To prevent this, a shift towards renewable energy sources
has been on the rise. Harnessing renewable energy sources such as wind,
solar, and hydroelectric will be needed if we are to reduce the harmful
air pollution from sources such as coal, oil, natural gas, and nuclear
energy. Most of these non-renewable sources come from the past so the
infrastructure has been in place for a long time. We need to build new
infrastructure for renewable energy sources such as building dams, wind
turbines, and solar panels. This will come at a financial cost which
will have to ultimately come from the citizens in the form of increased
taxes or increased electricity payments.

From the CES 2020 dataset, we will analyze whether political preference,
education level, and type of area a person is living in (family income?)
affects their stance on reducing pollution and climate change
prevention.

The scenario that was proposed in the survey was: Require that each
state use a minimum amount of renewable fuels (wind, solar, and
hydroelectric) in the generation of electricity even if electricity
prices increase a little. And the possible responses were: support or
oppose. We can consider this support of or opposition of this scenario
as their stance on the necessity of reducing pollution and climate
change prevention.

There will always be a cost associated with climate change. Thus it is
important to recognize that certain sacrifices have to be made in order
to benefit our earth.

\hypertarget{sec-data}{%
\section{Data}\label{sec-data}}

\hypertarget{model}{%
\section{Model}\label{model}}

We want to model an American's stance on climate change, specifically
their position on requiring that each state use a minimum amount of
renewable fuels in the generation of electricity even in electricity
prices increase a little. In our model, we consider a person's political
preference, education level, and type of area they are living in to
predict their stance on this topic.

\begin{align} 
y_i|\pi_i &\sim \mbox{Bern}(\pi_i) \\
\mbox{logit}(\pi_i) &= \alpha + \beta \times \mbox{household_income}_i + \gamma \times \mbox{education}_i + \delta \times \mbox{livingArea}_i\\
\alpha &\sim \mbox{Normal}(0, 2.5) \\
\beta &\sim \mbox{Normal}(0, 2.5) \\
\gamma &\sim \mbox{Normal}(0, 2.5) \\
\delta &\sim \mbox{Normal}(0, 2.5)
\end{align}

Where:

\begin{itemize}
\tightlist
\item
  \(y_i\) is the binary outcome variable, representing
\item
  \(\pi_i\) is the probability that respondent
\item
  \({household_income}_i\) is a predictor variable, representing the
  household income of the respondent \(i\),
\item
  \({education}_i\) is a predictor variable, representing the education
  level of the respondent
\item
  \({livingArea}_i\) is a predictor variable, the residential area that
  the respondent is living in (urban, rural, suburban, etc.) \(i\).
\end{itemize}

\hypertarget{model-set-up}{%
\subsection{Model set-up}\label{model-set-up}}

\hypertarget{model-justification}{%
\subsubsection{Model justification}\label{model-justification}}

\hypertarget{results}{%
\section{Results}\label{results}}

\hypertarget{tbl-model-summary-table}{}
\begin{table}
\caption{\label{tbl-model-summary-table}Explanatory models of renewable energy support based on household
income, education level, and land type of residence (n = 3000) }\tabularnewline

\centering
\begin{tabular}[t]{lc}
\toprule
  & Support renewable energy even at a higher monetary cost\\
\midrule
(Intercept) & \num{-0.802}\\
 & \vphantom{1} (\num{0.286})\\
household\_income\$10,000 - \$19,999 & \num{-0.235}\\
 & (\num{0.226})\\
household\_income\$20,000 - \$29,999 & \num{-0.195}\\
 & (\num{0.205})\\
household\_income\$30,000 - \$39,999 & \num{-0.004}\\
 & (\num{0.212})\\
household\_income\$40,000 - \$49,999 & \num{-0.143}\\
 & (\num{0.213})\\
household\_income\$50,000 - \$59,999 & \num{-0.245}\\
 & (\num{0.217})\\
household\_income\$60,000 - \$69,999 & \num{0.022}\\
 & (\num{0.227})\\
household\_income\$70,000 - \$79,999 & \num{0.208}\\
 & (\num{0.218})\\
household\_income\$80,000 - \$99,999 & \num{0.135}\\
 & (\num{0.215})\\
household\_income\$100,000 - \$119,999 & \num{0.207}\\
 & (\num{0.233})\\
household\_income\$120,000 - \$149,999 & \num{0.134}\\
 & (\num{0.234})\\
household\_income\$150,000 - \$199,999 & \num{0.127}\\
 & (\num{0.259})\\
household\_income\$200,000 - \$249,999 & \num{0.317}\\
 & (\num{0.325})\\
household\_income\$250,000 - \$349,999 & \num{0.265}\\
 & (\num{0.417})\\
household\_income\$350,000 - \$499,999 & \num{0.424}\\
 & (\num{0.514})\\
household\_income\$500,000 or more & \num{0.465}\\
 & (\num{0.579})\\
educationHigh school graduate & \num{0.031}\\
 & (\num{0.266})\\
educationSome college & \num{-0.189}\\
 & (\num{0.264})\\
education2-year & \num{-0.140}\\
 & (\num{0.281})\\
education4-year & \num{-0.620}\\
 & (\num{0.273})\\
educationPost-grad & \num{-0.691}\\
 & (\num{0.286})\\
living\_areaSuburb & \num{0.274}\\
 & (\num{0.108})\\
living\_areaTown & \num{0.535}\\
 & (\num{0.133})\\
living\_areaRural & \num{0.711}\\
 & (\num{0.114})\\
\midrule
Num.Obs. & \num{3000}\\
R2 & \num{0.040}\\
Log.Lik. & \num{-1832.708}\\
ELPD & \num{-1857.3}\\
ELPD s.e. & \num{21.2}\\
LOOIC & \num{3714.6}\\
LOOIC s.e. & \num{42.5}\\
WAIC & \num{3714.5}\\
RMSE & \num{0.46}\\
\bottomrule
\end{tabular}
\end{table}

We performed logistic regression analysis on \_\_\_ observations of the
total \_\_\_ observations in the cleaned dataset. We are interested in
whether or not a person's political preference, household income,
education, or area where they are living in will impact their stance
towards preventing climate change.

\hypertarget{discussion}{%
\section{Discussion}\label{discussion}}

\hypertarget{sec-first-point}{%
\subsection{First discussion point}\label{sec-first-point}}

If my paper were 10 pages, then should be be at least 2.5 pages. The
discussion is a chance to show off what you know and what you learnt
from all this.

\hypertarget{second-discussion-point}{%
\subsection{Second discussion point}\label{second-discussion-point}}

\hypertarget{third-discussion-point}{%
\subsection{Third discussion point}\label{third-discussion-point}}

\hypertarget{weaknesses-and-next-steps}{%
\subsection{Weaknesses and next steps}\label{weaknesses-and-next-steps}}

Weakness in the question. It states ``\ldots{} even if electricity
prices increase a little'' but this is can mean something different to
everyone. The wording leaves it up to the reader for interpretation on
how much the price will increase. Thus, a person's decision may be
different depending on if this question had a more concrete wording with
a price range or percent.

\newpage

\appendix

\hypertarget{appendix}{%
\section*{Appendix}\label{appendix}}
\addcontentsline{toc}{section}{Appendix}

\hypertarget{additional-data-details}{%
\section{Additional data details}\label{additional-data-details}}

\newpage

\hypertarget{references}{%
\section{References}\label{references}}



\end{document}
